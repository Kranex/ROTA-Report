\section{Program Code}
\par The code of ROTA is split into three seperate programs.
\begin{itemize}
  \item The controller - Written in Java. https://github.com/Kranex/ROTA-Controller
  \item The Server - Written in C. https://github.com/Kranex/ROTA-Server
  \item The Interpreter - Written in Arduino C/C++. https://github.com/NotSoScientific/ROTA-Interpreter
\end{itemize}
In the interest of saving trees, the 700+ lines of code are not included in this document,
if you wish to find them please use the links above to access them. A description of
each program, however, is provided below.

\subsection{The Controller}
\par
The controller was written in Java with the help of the Android Studio IDE.
The main function of the controller is to create two floating joysticks to allow
users to submit input to ROTA. It is essentially an Xbox or PlayStation controller
but without any buttons.
One joystick is created on the left and one on the right, they will place themselves
under the first touch on their half of the screen and the centre of the joystick
will remain at that point until the finger is lifted again. This allows for some degree of personal adjustability; not everyone holds their phone the same way,
and not all phones are the same size or shape, so the joysticks should not be fixed
to one place on the screen.

\par
The controller also tries to connect to a TCP socket on the IP address 192.168.12.1
port 55455, the Raspberry Pi was set up to maintain a consistent gateway address
so that the controller would never have to ask for the IP address of the server,
the server was also configured to only use port 55455.
Once a connection has been created the controller will send an update packet every
few miliseconds. This packet consists of 5 bytes \{a,x1,y1,x2,y2\}.
\begin{enumerate}
  \item 'a' is an undefined byte. It can be used for anything; it was added in case
   some kind of status update or emergency stop button was introduced.
  \item x1 and y1 are the influences of the left joystick. Their value will be
  between -127 to 127. (0,0) would mean the joystick is centered. (0,127) would mean
  the joystick was centered on the x-axis but pushed all the way up on the y-axis.
  \item x2 and y2 are just the same as x1 and y1 but are for the right joystick.
\end{enumerate}
The packet does not require any leading or ending byte as TCP guarantees that all
packets will arrive and they will all arrive in the correct order. Thus, on the server,
if the Server reads 5 bytes it will always read \{a,x1,y1,x2,y2\} if a packet exists to be read.

\subsection{The Server}
\par The server is by far the most convoluted program of the three, which is thanks
to it having been programmed in C.
It does only one thing:
 forward all packets received from the controller, straight to the Arduino over serial.
It does, however, add 127 to each of the signed bytes as the serial connection
seemed to really dislike the signed bytes. 127 is then immediately subtracted from
the byte when it gets to the Arduino.
This is all handled in about 5 lines of code, however, there are nearly 300 lines
surrounding it most of which is simply initialisation for the TCP server and serial
connection.
On the Raspberry Pi, the server is added to the list of programs to start at boot
with systemd, thus no login is required, the Raspberry Pi must only be booted for
the server to be created.

\subsection{The Interpreter}
\par The interpreter, for lack of a better name, is the program running on the Arduino
which `interprets' the x and y influences recieved from the server and translates them
into motor speed, servo angle or whatever else we happen to connect to ROTA.
This job could have been done with the Raspberry Pi alone, however, we would then have
had the constraints of few GPIO to communicate with, as well as the lower voltage of
the GPIO (3.3V instead of 5V) to deal with. Thus we decided to use the Arduino as the
step between the Raspberry Pi and the extra peripherals.
\par After initialising the serial connections and any global variables, the arduino then
reads a single byte from the serial buffer and acts upon it. It keeps count of which particular
byte it is reading and then, as we have it currently set up, if it is the left y, uses
it to set the motor speed and direction or if it is the right x, use it to set the
servo angle. The particular stick or axis that controls each of these can easily be changed,
the only reason we have it set up as so is that it made practical sense to split
the controls of throttle and steering on to separate sticks. As for why the sticks
are capable of moving in both axis when only one is required, it doesn't impair their
function, it was easier to make two identical joysticks over two unique ones and
most importantly, if we had the time and the servos to do it, we would have like to
have added a moveable camera. Unfortunately, this was not the case, but now at least
it features the ability to have such a camera with the addition of only two if statements.
