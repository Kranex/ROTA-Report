\section{Evaluation}
\par Throughout the development of ROTA there have been things that needed to be tested.
After all, there are three devices that could fail. The first thing that was programmed was
the app, after fixing many small issues that showed up throughout development, three major
problems showed.  
\begin{enumerate}
  \item If the app is disconnected from the server, by losing the wifi or something
  similar, it would never reconnect. To fix this, the client initialisation thread was
  extended to perpetually loop, if the client was not connected, it would try to reconnect it,
  otherwise, it would send an update packet. This also helped limit the number of packets
  being sent to the server as originally a packet would be sent every time the user
  stimulated the touch screen, this helped prevent the Arduino from becomming overloaded.
  \item If the screen focus shifts, causing the app to leave fullscreen, it never reverts.
  This does not impact usability, however it is annoying. This was fixed by giving the
  Android immersive environment the sticky flag. Now, popups and dragging the system bar into
  view does not resize the app as they are rendered 'over' the app. When they disappear
  the app is still fullscreen.
  \item If the user leaves the app, Android will `suspend' it. This is great, however,
  it wreaks havoc with the server connection. Very strange things can happen, the best
  you can hope for is that the app crashes, at worst, the app tries to reistablish the
  connection and you end up with 5 or 6 rogue threads spamming TCP Requests.
  To prevent this, the app was set up to destroy itself completely when minimised. This
  loses no functionality, however, as the app would have to be restarted when it comes
  out of suspension anyway, it is just easier to restart from the beginning of the program.
\end{enumerate}
There were never really any issues with the server, par some minor confusion caused by
C's lack of a 'byte' type and chars being unsigned on some systems but not others.
This was easily fixed by declaring all chars, being used as bytes, as signed chars.
\par
Another problem came with the Arduino, it would not accept signed bytes across serial.
this meant that either the transmission method needed to be revised or that unsigned bytes
needed to be sent instead. As all bytes were between -127 and 127, adding 127 before sending
and subtracting it after receiving solved this problem.
\par
The biggest problem of them all was how to supply power to the Arduino,
the Raspberry Pi, and the motor shield. In the end, it was decided the Raspberry Pi and
the motor shield would receive power from a large USB power bank. The Raspberry Pi would
then power the Arduino. This seemed to work on the bench when the Arduino was plugged into
the PC for tuning the servo and running light tests on the motor. However, the Raspberry Pi
could not deliver the current to the Arduino to power the servo and the logic side of the
motor controller, this would crash the Raspberry Pi and all hell would break loose.
To counter this issue the Arduino was given its very own 9v battery. While this is not
perfect, it does seem to solve the problem. Ideally a dedicated Lithium Polymer
battery would have been preferred, a 2200mah 7.4v LiPo would be small and light enough
to fit on the frame, however, they are not exactly cheap and the Raspberry Pi would require
something extra to step the voltage down to 5v.
\par
After the power problem was sorted, the car worked almost perfectly, it can pick up quite
some speed, every now and then there is a slight delay in the controls however not enough
to cause serious issues. The exact reason for the delay is not understood, amongst other things
it could be due to the Arduino being flooded with more packets than it can handle, or it
could be because the Raspberry Pi suffers from a similar problem. The only minor faults are
that the servo is rather poorly attached to the steering, allowing for a small margin of
free movement. This means that the steering is never exactly centered and the car needs
constant steering adjustments to stay on course. It was also noticed that the front wheels
would lock on the frame when turning. This was because all the weight had crushed the suspension.
To fix this the wheel arches were removed.
